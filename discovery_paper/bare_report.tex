
\documentclass[11pt,journal]{discovery_paper}





% *** CITATION PACKAGES ***
\usepackage{cite}


% *** GRAPHICS RELATED PACKAGES ***
\ifCLASSINFOpdf
  % \usepackage[pdftex]{graphicx}
\else
  % \usepackage[dvips]{graphicx}
\fi


% *** MATH PACKAGES ***
\usepackage{amsmath}


% *** SPECIALIZED LIST PACKAGES ***
\usepackage{algorithmic}


% *** ALIGNMENT PACKAGES ***
\usepackage{array}


% *** SUBFIGURE PACKAGES ***
%\ifCLASSOPTIONcompsoc
%  \usepackage[caption=false,font=normalsize,labelfont=sf,textfont=sf]{subfig}
%\else
%  \usepackage[caption=false,font=footnotesize]{subfig}
%\fi


% *** FLOAT PACKAGES ***
\usepackage{fixltx2e}

\usepackage{stfloats}

\usepackage{dblfloatfix}


%\ifCLASSOPTIONcaptionsoff
%  \usepackage[nomarkers]{endfloat}
% \let\MYoriglatexcaption\caption
% \renewcommand{\caption}[2][\relax]{\MYoriglatexcaption[#2]{#2}}
%\fi


% *** PDF, URL AND HYPERLINK PACKAGES ***
\usepackage{url}


\begin{document}

\title{Evidence for the Existence of the Higgs Boson (or smth like that)}

\author{Daniel Gaivao Lozano, Dillen Lee, Callum McFadyen, Samuel Tsang, Kieren Ventham}

\markboth{5}%
{Shell \MakeLowercase{\textit{et al.}}:}


\maketitle

\begin{abstract}
Strong evidence for the existence of a particle with a mass of 125 GeV is discovered.
\end{abstract}


\section{Introduction}
This section contains a brief overview of the procedures that led to the identification of a new particle. It may also be useful to include the motivations and some extra context about the discovery in this section.


\section{Data Generation and Parameterisation}
Here you can include information about the simulated dataset and the parameterisation of the distribution. Remember to include graphs and figures that you have generated, and elaborate on them


\section{Hypothesis Testing}
Here is where the different hypotheses tests carried out are explained (i.e. background-only or background and signal) and the outcomes are shown.  This section can also include topics like examining the ‘goodness of fit’, with the reduced ${\chi}^2$ statistics.


\section{Results and Analysis}
The discussion of the results obtained, along with explanations and analysis and statistical inferences go here. The errors and uncertainties on your calculations can also be mentioned here.


\section{Conclusion}
Your chance to succinctly explain how and why the process outlined above, led to and confirms the discovery.



\section{References}
Correctly cite any sources of external supporting material here.

%Below is an example figure added
%\begin{figure}[h!]
%\centering
%\includegraphics[width=2.5in]{myfigure}
%Having the image saved in the file directory will display it here.
%\caption{This is the caption, to elaborate on the figure included.}
%\label{fig_sim}
%\end{figure}


\appendices
\section{Interesting Extra Information}
Any information included here is optional for the reader and should not be relied upon in order to understand the report in general.

\section*{Acknowledgement}
Acknowledge any individuals, groups or organisations who are not team members who have supported you with this work


\bibliographystyle{ieeetran}



\end{document}
